\documentclass[a4paper]{article}
\usepackage{fullpage}
\usepackage{amsmath}
\usepackage{ulem}
\newcommand{\ra}{$\rightarrow$}
\newcommand{\fa}{$\forall$}
\newcommand{\te}{$\exists$}
\newcommand{\n}{$\not$}
\newcommand{\bi}{\begin{itemize}}
\newcommand{\ei}{\end{itemize}}
\newcommand{\be}{\begin{enumerate}}
\newcommand{\ee}{\end{enumerate}}
\newcommand{\claim}{\underline{claim:}}
\newcommand{\prf}{\underline{Prf:}}
\newcommand{\contradiction}{\textbf{contradiction:}}
\begin{document}
  \section{info}
	Prof: Nicole Immorlica\\
	3 Quizzes: Oct 11, Nov 3, Dec 1\\
	Psets: assigned weekly, due on week w/o quiz, should take 6-8 hours\\
	Textbook is recommended, NOT required\\
	Assigned reading from MIT Open Courseware\\
	Weekly mini-quizzes\\
	Grading:
	\begin{itemize}
	  \item 60\% quizzes (20\% each)
	  \item 40\% psets (5\% each)
      \item 7\% chellenge problems (optional extra credit)
      \item 8\% mini quizzes (1\% each)
	\end{itemize}
	Grades:
	\begin{itemize}
	  \item $\ge$ 94\% A
	  \item $\ge$ 88\% A-
	  \item ...
	  \item $\ge$ 64\% C-
	\end{itemize}
	Exams curved so avg = 75\%\\
	Website http://users.eecs.northwestern.edu/~nickle/310/2011/\\
  \section{9.20.11}
    \subsection{Graph Coloring}
      \begin{enumerate}
        \item Color a map - color map so that no two adjacent areas are the same color
        \item Schedule exams - no exam conflicts with minimum number of hours
      \end{enumerate}
      Abstraction: use graphs\\
      \begin{itemize}
        \item make all items points on graph
        \item map example: states A B C D, connect states that touch (i.e. conflict)
        \item exam example: classes A B C D, connect exams that are at same time (conflict)
        \item A: red, B: yellow, D: blue \ra C: yellow (does not conflict with B) 
      \end{itemize}
    \subsection{Tiling}
      E.g. 4x4 tile grid with UR and LL tile missing. tiles are 1x2, can we cover the area?\\
      \begin{itemize}
        \item sanity check: even number of tiles?
          \subitem yes, 14
        \item start with smaller analogy
          \subitem can we cover 2x2?
          \subitem no, tiles are too big
          \subitem chessboard analogy: alternating pattern \ra more white squares than black
          \subitem \ra cannot be done
      \end{itemize}
      E.g. game of chomp. Alice and Bob eat mxn chocolate bar. \\
      \textbf{important method: reduction to simpler case}
    \subsection{logic}
      \subsubsection{Proposition}
        statement with truth value
      \subsubsection{Operators}
        not, or, and, xor, implies ( \ra ), iff\\
        ~q \ra ~p $\equiv$ p\ra q
      \subsubsection{Predicate}
        proposition who's truth value depends on variable\\
        E.g. P(n) = n is divisible by 3
      \subsubsection{Quantifiers}
        \te there exists\\
        \fa for all\\
        \fa integers n \te integer m: P(n,m), and other combinations...\\
        \ra order of quanitifiers \textbf{matters}
      \subsubsection{negating quantifiers}
        negate 'it is not the case that all integers are even'
        \begin{itemize}
          \item Even(n) = n is even
          \item chunk statement into logic symbols
            \subsubitem \fa integers n: Even(n)
            \subsubitem \n (\fa integers n: Even(n)) = \te an integer n: \n Even(n)
        \end{itemize}
        similarily, \n (\te ineger n: P(n)) \ra \fa integer n: \n P(n)\\
      \subsubsection{Negating Mixtures}
        \n (\fa integer n: \te integer m: P(n,m)) \ra \te integer n: \n (\te integer m: P(n,m)) \ra \te integeter n: \fa integer m: \n P(n,m)\\
        E.g. \textbf{Goldbach's Conjecture} Every even integer $>$ 2 can be written as a sum of 2 primes\\
        \begin{itemize}
          \item let Even be the set of even primes $>$ 2
          \item \fa n in Even, \te p in Primes, \te q in Primes: n = p+q
        \ei
  \section{9.22.11}
    Cover some stuff we talked about yesterday, quantifiers blablabla\\
    % some guy is being a dumbass...
    \subsection{Proofs}
      \subsubsection{Proof guideline I: Direct Proof}
        to prove p \ra q\\
        \begin{itemize}
          \item assume p is true
          \item conclude that q is true via logical deductions
        \end{itemize}
      \subsubsection{Proof guideline II: Proof by contrapositive}
        to prove p \ra q
        \begin{itemize}
          \item prove that \n q \ra \n p
        \end{itemize}
      \subsubsection{Proof guideline III: Proof by contradiction}
        to prove p
        \begin{itemize}
          \item assume \n p
          \item derive a contradiction with assumption
        \end{itemize}
      \subsubsection{Proof guideline IV: Proof by cases}
        to prove p
        \begin{itemize}
          \item break p into cases
          \item prove each case individually
        \end{itemize}
    \subsection{Examples of proofs}
      facts about integers:
      \begin{itemize}
        \item n is even $\equiv$ n=2k
        \item n is odd $\equiv$ n = 2k + 1
        \item d divides n (d|n) $\equiv$ n=ld
        \item p is prime $\equiv$ the only divisors 1 and p
      \end{itemize}
      \subsubsection{Ex. 1 (direct proof)}
        \underline{claim:} if x is odd, then x + 1 is even\\
        \underline{Prf:} by direct proof
        \begin{itemize}
          \item p: x is odd
          \item q: x + 1 is even
          \item[]
          \item assume x is odd
          \item then, x = 2k + 1
          \item then, x + 1 = 2k + 1 + 1
          \item x + 1 = 2(k + 1)
          \item \ra q is even
        \end{itemize}
      \subsubsection{Ex.2 (contrapositive proof)}
        \underline{claim:} let d|n. If n is odd, then d is odd\\
        \underline{Prf:} by contrapositive
        \begin{itemize}
          \item p: n is odd
          \item q: d is odd
          \item[]
          \item we show that \n q \ra \n p
            \subitem let d|n, then if d is even, n is even
          \item assume d|n and d is even
          \item then d = 2k and n=ld so n = l(2k) so n = 2(lk)
          \item \ra n is even
        \end{itemize}
      \subsubsection{Ex. 3 (proof by contradiction)}
        game of chomp\\
        \underline{claim:} alice wins in chomp\\
        \underline{Prf:} by contradiction
        \begin{itemize}
          \item assume alice loses
          \item for any first move of alice, there is a move for bob such that bob
          can force a win
          \item consider move (m,n) for alice, and suppose bob moves (i,j)
          \item by assumption bob has a sequence of responses for any moves of
          alice such that he wins
          \item but then alice could have chosen (i,j) as her first move and
          copied bobs sequence of responses
          \item \textbf{Contradiction:} by copying bob alice won
          \item \ra alice wins
        \end{itemize}
      \subsubsection{Ex. 4 (proof by contradiction)}
        \underline{claim:} there is no largest even integer\\
        \underline{Prf:} by contradiction
        \begin{itemize}
          \item assume \te greatest even integer N
          \item then N=2k
          \item so then N+2 = 2k+2
          \item so N+2 is even
          \item \textbf{contradiction:} N+2>N
        \end{itemize}
      \subsubsection{Ex. 5 (proof by cases)}
        \underline{claim:} \fa n, m, $\frac{n+m}{2} \le max(n,m)$\\
        \underline{Prf:} by cases
        \begin{enumerate}
          \item n=m
          \item n<m
          \begin{itemize}
            \item first rewrite n+m $\le$ 2max(n,m)
            \item then n+m < m+m = 2m = 2max(n,m)
          \end{itemize}
          \item n>m
        \end{enumerate}
      \subsubsection{Ex. 6 (proof by contradiction)}
        \underline{claim:} there are infinitely many primes\\
        \underline{Prf:} by contradiction
        \begin{itemize}
          \item assume there are finitely many primes
          \item call them $p_{1},p_2,...,p_{n}$
          \item consider $q=p_{1}p_2...p_{n}+1$
          \item then either q is prime in which case we're done, or \te prime
          divisor p
          \item so p = $p_{i}$ for some i
          \item sp p|$p_{1}p_2...p_{n}$ and p|q, so p|(q-($p_{1}...p_{n}$))
          \item so p|1
        \end{itemize}
      \subsubsection{Ex. 7}
        \underline{claim:} $\sqrt{2}$ is irrational\\
        (recall a number x is rational if \te integers p,q : x=$\frac{p}{q}$)\\
        \underline{Prf:} by contradiction
        \begin{itemize}
          \item $\sqrt{2}$ is rational
          \item then $\sqrt{2}= \frac{p}{q}$ s.t. they have no common divisor
          \item $p^2=2q^2$, so $p^2$ is even
          \item \textbf{Lemma:} if $p^2$ is even, then p is even
          \item assume p is even, so p=2k
          \item so $2q^2 = (2k)^2$, so $q^2$ is even
          \item so q is even
          \item \textbf{contradiction:} if p and q are even, they have a common
          divisor
        \end{itemize}
      \subsubsection{Ex. 8}
        \underline{claim:} if a number n's digits add up to a number divisible by 9,
        then 9|n\\
        \underline{Prf:} by direct proof
        \begin{itemize}
          \item assume the sum of digits is divisible by 9
          \item n = 3547 = 7*1 + 4*10 + 5*100 + 3*1000
          \item $n = nx_010^0 + x_110^1 + ... + x_n10^n$
          \item so 9|($x_0 + ... + x_n$)
          \item so n = ($xA_0 + ... + x_n$) + $9x_1 + 9*9x_2 + ... + 9*9..9*9x_n$
          \item 9 divides each term, so 9|n!
        \end{itemize}
      \subsubsection{Ex. 9}
        \claim \te irrational numbers x,y : $x^y$ is rational\\
        \prf by cases
        \bi
          \item try $x=y=\sqrt{2}$
          \item doesnt work
          \item try $x=\sqrt{2}^{\sqrt{2}} , y=\sqrt{2}$
          \item $x^y = 2$
          \item MIND. MOTHERFUCKING. BLOWN
        \ei
\end{document}
